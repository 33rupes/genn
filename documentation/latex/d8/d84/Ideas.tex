\hypertarget{Ideas_Ideas}{}\section{for Ge\+N\+N development\+:}\label{Ideas_Ideas}

\begin{DoxyItemize}
\item add error checking in add\+Neuron\+Populations\+:
\begin{DoxyItemize}
\item maximum number of neuron populations? Or estimating resources beyound what is done at code generation stage?
\item check for missing minimal 1 neuron population?
\item check for name equality of neuron populations and throw error message?
\end{DoxyItemize}
\item add error checking for add\+Synapse\+Populations\+:
\begin{DoxyItemize}
\item maximal number / resource checking?
\item minimal number check -\/ revisit, should 0 be allowed?
\item check whether neuron populations exist.
\item check for unique name
\end{DoxyItemize}
\item There is a global flag int optimise\+Block\+Size= 1; this should be passed through to users or replaced by some approriate logic. This goes together with improving block size optimisation in my mind. (\hyperlink{global_8h}{global.\+h})
\item Is the macro for the number of predefined neuron types (N\+T\+Y\+P\+E\+N\+O in \hyperlink{modelSpec_8h}{model\+Spec.\+h}) really necessary?
\item \#define S\+P\+K\+\_\+\+T\+H\+R\+E\+S\+H 0.\+0f (in \hyperlink{modelSpec_8h}{model\+Spec.\+h}) is this really the way to do it? User control? global settings somewhere?
\item in \hyperlink{modelSpec_8cc}{model\+Spec.\+cc}\+: void N\+Nmodel\+::init\+Derived\+Neuron\+Para(unsigned int i)
\begin{DoxyItemize}
\item error checking for if it is called twice O\+R allowing multiple
\item calls but setting D\+N\+P values by index instead of simply appending.
\end{DoxyItemize}
\item Ge\+N\+N currently only supports parameter-\/homogeneous populations of neurons. An interesting extension could be individual parameters (this is a major change though, as parameters would need to become variables in the C\+U\+D\+A angle of things (and stored in device memory?).
\item Omissions in code documentation\+: \hyperlink{modelSpec_8cc}{Model\+Spec.\+cc} (does it need more details?)
\item transition synapses from explicit standard models supported during code generation to more generic type that would allo user additions similar to the n\+Models vector for neurons
\item If synapses are defined in a n\+Synapses vector, remove the explicit parameter numbers etc from \hyperlink{modelSpec_8h}{model\+Spec.\+h}.
\item If used through the B\+R\+I\+A\+N interface there is a lot of unnecessary computation that the compiler might not pick up in its own optimisations. E.\+g. when using exponential Euler, B\+R\+I\+A\+N produces something like\+: 
\begin{DoxyCode}
lV * exp(-(\hyperlink{tmp_2model_2MBody__userdef_8cc_a943f07034774ef1261d62cd0d3d1fec9}{DT}) / 0.01) - 40.0 * 0.001 + 40.0 * 0.001 * exp(-(\hyperlink{tmp_2model_2MBody__userdef_8cc_a943f07034774ef1261d62cd0d3d1fec9}{DT}) / 0.01);
\end{DoxyCode}
 Here, D\+T is known at compile time, and hence
\begin{DoxyCode}
exp(-(\hyperlink{tmp_2model_2MBody__userdef_8cc_a943f07034774ef1261d62cd0d3d1fec9}{DT}) / 0.01) 
\end{DoxyCode}
 could be a dependent parameter and fully explicit, as could be the entire
\begin{DoxyCode}
- 40.0 * 0.001 + 40.0 * 0.001 * exp(-(\hyperlink{tmp_2model_2MBody__userdef_8cc_a943f07034774ef1261d62cd0d3d1fec9}{DT}) / 0.01)
\end{DoxyCode}
. We need to make an additional optimising step that identifies dependent parameters automatically. This could be a part of the genn device.\+py in the interface. Direct users of Ge\+N\+N are responsible for their own designs.
\end{DoxyItemize}

-\/\+Make use of C\+U\+R\+A\+N\+D possible for Poisson neurons 