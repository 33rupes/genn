Ge\+N\+N comes with a number of complete examples. At the moment, there are seven such example projects provided with Ge\+N\+N.

~\newline
\hypertarget{Examples_sec_Ex_OneComp}{}\section{Single compartment Izhikevich neuron(s)}\label{Examples_sec_Ex_OneComp}
This is a minimal example, with only one neuron population (with more or less neurons depending on the command line, but without any synapses). The neurons are Izhikevich neurons \cite{izhikevich2003simple} with homogeneous parameters across the neuron population. The model can be used by navigating to the {\ttfamily userproject/\+One\+Comp\+\_\+project} directory and entering a command line 
\begin{DoxyCode}
./generate\_run [\hyperlink{modelSpec_8h_ad703205f9a4d4bb6af9c25257c23ce6d}{CPU}/\hyperlink{modelSpec_8h_a39cb9803524b6f3b783344b2f89867b4}{GPU}] [n] [DIR] [EXE] [MODEL] [DEBUG OFF/ON].
\end{DoxyCode}
 This would create {\ttfamily n} tonic spiking Izhikevich neuron(s) with no connectivity, receiving a constant, identical 4 n\+A input current.

For example, navigate to the {\ttfamily userproject/\+One\+Comp\+\_\+project} directory and type 
\begin{DoxyCode}
make all
./generate\_run 1 1 Outdir OneComp\_sim OneComp 0.
\end{DoxyCode}
 to model a single neuron which output will be saved in the {\ttfamily Outdir\+\_\+output} directory.

~\newline
\hypertarget{Examples_sec_ex_mbody}{}\section{Insect Olfaction Model}\label{Examples_sec_ex_mbody}
This project implements the insect olfaction model by Nowotny et al. \cite{nowotny2005self} that demonstrates self-\/organized clustering of odours in a simulation of the insect antennal lobe and mushroom body. As provided the model works with conductance based Hodgkin-\/\+Huxley neurons \cite{Traub1991} and several different synapse types, conductance based (but pulse-\/coupled) excitatory synapses, graded inhibitory synapses and synapses with a type of S\+T\+D\+P rule.

To explore the model navigate to {\ttfamily userproject/\+M\+Body1\+\_\+project/} and type 
\begin{DoxyCode}
make all
./generate\_run 
\end{DoxyCode}
 This will show you the command line parameters that are needed, 
\begin{DoxyCode}
tools/generate\_run [\hyperlink{modelSpec_8h_ad703205f9a4d4bb6af9c25257c23ce6d}{CPU}/\hyperlink{modelSpec_8h_a39cb9803524b6f3b783344b2f89867b4}{GPU}] [#AL] [#KC] [#LHI] [#DN] [gscale] [DIR] [EXE] [MODEL] [DEBUG OFF/ON]
\end{DoxyCode}
 The tool generate\+\_\+run will generate connectivity files for the model {\ttfamily M\+Body1}, compile this model for the C\+P\+U and G\+P\+U and execute it. The command line parameters are the numbers of neurons in the different neuropils of the model and an overall synaptic strength scaling factor. A typical call would be, e.\+g., 
\begin{DoxyCode}
../../tools/generate\_run 1 100 1000 20 100 0.0025 test1 MBody1 0
\end{DoxyCode}
 which would generate a model, and run it on the G\+P\+U (first parameter), with 100 antennal lobe neurons, 1000 mushroom body Kenyon cells, 20 lateral horn interneurons and 100 mushroom body output neurons. All output files will be prefixed with {\ttfamily test1} and stored in {\ttfamily test1\+\_\+output}. The model that is run is defined in {\ttfamily model/\+M\+Body1.\+cc} and debugging is switched off (the final command line parameter of 0).

As provided, the model outputs a file {\ttfamily test1.\+out.\+st} that contains the spiking activity observed in the simulation, where there are two columns in this A\+S\+C\+I\+I file, the first one containing the time of a spike and the second one the I\+D of the neuron that spiked. Users of matlab can use the scripts in the {\ttfamily matlab} directory to plot the results of a simulation. For more about the model itself and the scientific insights gained from it see \cite{nowotny2005self}.

~\newline
\hypertarget{Examples_sec_ex_poissonizh}{}\section{Izhikevich Network Driven by Poisson Input Spike Trains\+:}\label{Examples_sec_ex_poissonizh}
This example project is inspired by Izhikevich's model of a prototype thalamo-\/cortical network \cite{izhikevich2003simple} with 80\% excitatory and 20\% inhibitory neurons, randomly connected with a given probability.

The model can be compiled by navigating to the {\ttfamily userproject\textbackslash{}Poisson\+Izh\+\_\+project} directory and typing 
\begin{DoxyCode}
make all
\end{DoxyCode}
 Subsequently it can be invoked using the following command line 
\begin{DoxyCode}
./generate\_run [\hyperlink{modelSpec_8h_ad703205f9a4d4bb6af9c25257c23ce6d}{CPU}/\hyperlink{modelSpec_8h_a39cb9803524b6f3b783344b2f89867b4}{GPU}] [#Poisson] [#Izhikevich] [pConn] [gscale] [DIR] [EXE] [MODEL] [DEBUG OFF/ON]
\end{DoxyCode}
 For example, navigate to the {\ttfamily userproject/\+Poisson\+Izh\+\_\+project} directory and type 
\begin{DoxyCode}
./generate\_run 1 100 10 0.5 2 Outdir PoissonIzh\_sim PoissonIzh 0
\end{DoxyCode}
 This will generate a network of 100 Poisson neurons connected to 10 Izhikevich neurons with a 0.\+5 probability. The same network with sparse connectivity can be used by addind the synapse population with sparse connectivity in \hyperlink{PoissonIzh_8cc}{Poisson\+Izh.\+cc} and by uncommenting the lines following the \char`\"{}//\+S\+P\+A\+R\+S\+E C\+O\+N\+N\+E\+C\+T\+I\+V\+I\+T\+Y\char`\"{} tag in Poisson\+Izh.\+cu.\hypertarget{Examples_sec_ex_izhnetwork}{}\section{Pulse-\/coupled Izhikevich network}\label{Examples_sec_ex_izhnetwork}
Can be used as\+: 
\begin{DoxyCode}
tools/generate\_run\_1comp generate\_izhikevich\_network\_run [\hyperlink{modelSpec_8h_ad703205f9a4d4bb6af9c25257c23ce6d}{CPU}/\hyperlink{modelSpec_8h_a39cb9803524b6f3b783344b2f89867b4}{GPU}] [#n] [#Conn] [gscale] [outdir] [
      executable name] [model name] [debug OFF/ON] [use previous connectivity OFF/ON]
\end{DoxyCode}
 This example creates a pulse-\/coupled network \cite{izhikevich2003simple} with 80\% excitatory 20\% inhibitory neurons, each connecting to \#\+Conn neurons with sparse connectivity.

To use an example, navigate to the \char`\"{}userproject/\+Izh\+\_\+\+Sparse\+\_\+project\char`\"{} directory and type 
\begin{DoxyCode}
../../tools/generate\_izhikevich\_network\_run 1 10000 1000 1 Outdir Izh\_sim\_sparse Izh\_sparse 0 0
\end{DoxyCode}
\hypertarget{Examples_sec_ex_izhdelay}{}\section{Izhikevich network with delayed synapses}\label{Examples_sec_ex_izhdelay}
This example project demonstrates the delayed synapse feature of Ge\+N\+N. It creates a network of three Izhikevich neuron groups, connected all-\/to-\/all with fast, medium and slow synapse groups. Neurons in the output group only spike if they are simultaneously innervated by the input neurons, via slow synapses, and the interneurons, via faster synapses.

To run this example project, cd to \char`\"{}\$\+G\+E\+N\+N\+\_\+\+P\+A\+T\+H/userproject/\+Syn\+Delay\+\_\+project\char`\"{} and type 
\begin{DoxyCode}
buildmodel \hyperlink{classSynDelay}{SynDelay} && make clean && make && ./bin/release/syn\_delay 1 output
\end{DoxyCode}


~\newline
 

 \hyperlink{Quickstart_sec}{Previous} $\vert$ \hyperlink{Examples_sec}{Top} $\vert$ \hyperlink{UserManual_sec}{Next} 